\documentclass{article}
\usepackage{amsmath}

\title{NUMT Risk Confidence Intervals}
\date{}

\begin{document}

\maketitle

\section*{Problem Setup}
This note documents the method used to calculate 95\% confidence intervals for the risk of mitochondrial diseases caused by nuclear gene mutations, as inferred from nuclear mitochondrial (NUMT) gene sequence data.

The problem setup is as follows:
\begin{itemize}
    \item You have \( k \) loci.
    \item For each locus \( i \), you have:
    \begin{itemize}
        \item \( n_i \) trials,
        \item \( s_i \) observed non-wild-type alleles,
        \item An estimated probability \( \hat{p}_i = \frac{s_i}{n_i} \).
    \end{itemize}
    \item The probability \( P \) of having double non-wild-type alleles at any of the \( k \) loci is (one minus the probability of being unaffected):
    \[
    P = 1 - \prod_{i=1}^{k} (1 - p_i^2)
    \]
\end{itemize}

We compute the confidence interval on the risk of having double non-wild-type alleles at any of $k$ nuclear genome loci using the delta method.

\section*{Step 1: Confidence Interval for \( p_i \)}

The Clopper-Pearson method can be used to construct an exact confidence interval for the binomial proportion \( p_i \). For each locus \( i \), the estimate \( \hat{p}_i \) is calculated as \( \hat{p}_i = \frac{s_i}{n_i} \). The confidence interval for \( p_i \) is given by:
\[
\text{CI}_{95\%}(p_i) = \left[ L_i, U_i \right]
\]
where \( L_i \) and \( U_i \) are the lower and upper bounds of the Clopper-Pearson confidence interval for \( p_i \) based on the observed successes \( s_i \) and trials \( n_i \).

\section*{Step 2: Confidence Interval for \( p_i^2 \)}

Given that the probability \( p_i \) lies within the interval \( \left[ L_i, U_i \right] \), the confidence interval for \( p_i^2 \) can be derived by squaring the endpoints:
\[
\text{CI}_{95\%}(p_i^2) = \left[ L_i^2, U_i^2 \right]
\]
This interval is valid because the squaring function is a continuous and monotonic bijection on the interval \( [0,1] \), preserving the confidence level.

\section*{Step 3: Compute the Probability \( P \)}

First compute the probability \( P \) using the estimated probabilities \( \hat{p}_i \):

\[
P = 1 - \prod_{i=1}^{k} (1 - \hat{p}_i^2)
\]

\section*{Step 4: Apply the Delta Method}

For each locus \( i \), compute the partial derivative of \( P \) with respect to \( \hat{p}_i \):
\[
\frac{\partial P}{\partial \hat{p}_i} = 2 \hat{p}_i \cdot \prod_{j \neq i} (1 - \hat{p}_j^2)
\]
This derivative reflects how \( P \) changes with respect to each \( \hat{p}_i \).

Approximate the variance of \( P \) by summing the contributions from each \( \hat{p}_i \):
\[
\text{Var}(P) \approx \sum_{i=1}^{k} \left(\frac{\partial P}{\partial \hat{p}_i}\right)^2 \cdot \text{Var}(\hat{p}_i)
\]
where
\[
\text{Var}(\hat{p}_i) = \frac{\hat{p}_i(1 - \hat{p}_i)}{n_i - 1}
\]

The standard deviation of \( P \) is the square root of the variance:
\[
\sigma_P = \sqrt{\text{Var}(P)}
\]

\section*{Step 5: Construct the Confidence Interval for \( P \)}

Assuming \( P \) follows a normal distribution (reasonable under the Central Limit \\
Theorem for large \( n_i \)), a \( 95\% \) confidence interval for \( P \) can be constructed as:

\[
\text{CI}_{95\%} = \left[ P - 1.96 \cdot \sigma_P, P + 1.96 \cdot \sigma_P \right]
\]

\end{document}